\documentclass[12pt]{article}
\usepackage[english]{babel}
\usepackage[utf8x]{inputenc}
\usepackage{amsmath}
\usepackage{graphicx}
\usepackage{caption}
\usepackage{subcaption}
\usepackage{etoolbox}
\usepackage{changepage}
\usepackage{titlesec}
\usepackage[parfill]{parskip}
\usepackage[margin=1in]{geometry}
\usepackage{times}
\usepackage[numbers,super]{natbib}
\usepackage{float}

\titleformat*{\section}{\normalsize\bfseries} % Makes section titles 12 pt font


%----------------------------------------------------------------------------------------
%  TITLE SECTION
%----------------------------------------------------------------------------------------
\title{\large \textbf{Twitter in the Engineering Classroom}} % using \large makes the title approximately 14 pt.
\author{\vspace{-5ex}}
\author{\normalsize Devin R. Berg\\
\normalsize bergdev@uwstout.edu\\
\normalsize Engineering and Technology Department\\\
\normalsize University of Wisconsin-Stout}
\date{\vspace{-5ex}} % This leaves the date blank.

\makeatletter % This gets the margins for the title set.
\patchcmd{\@maketitle}{\begin{center}}{\begin{adjustwidth}{0.5in}{0.5in}\begin{center}}{}{}
\patchcmd{\@maketitle}{\end{center}}{\end{center}\end{adjustwidth}}{}{}
\makeatother

\floatstyle{boxed}
\newfloat{textbox}{htbp}{lop}
\floatname{textbox}{Textbox}

%----------------------------------------------------------------------------------------

\begin{document}
\raggedright
\maketitle
\thispagestyle{empty}
\pagestyle{empty}

%----------------------------------------------------------------------------------------
%  PAPER CONTENTS
%----------------------------------------------------------------------------------------
\section*{Abstract}
The micro-blogging platform, Twitter, has been employed by some in higher education as a tool for enhanced student engagement. This platform has shown promise as an educational tool for the promotion of critical reading and writing and concise expression of ideas. However, it is unclear in what settings and under what circumstances Twitter can be effectively employed in the engineering classroom. These questions were explored over a multi-semester study of student participation in directed social media discussions within the engineering classroom. The various cohorts of students included in this study were drawn from engineering courses. Comparisons will be made between these multiple cohorts on the basis of active engagement in the assigned tasks, course performance, concept inventory performance, and student perception of the tasks. Through the process of using this practice in the classroom, it was found that there was difficulty encouraging engineering students to participate in Twitter discussions regardless of the incentive provided. Limited evidence was found of greater course achievement correlating with greater participation in Twitter based tasks. It is expected that greater effort is required in familiarizing students with the Twitter platform and increasing their comfort level with asking questions and carrying out discussions in a public forum.


\section*{Introduction}
The use of social media (SM) in the higher education classroom has expanded in recent years as educators come to realize the benefits of the various SM platforms for use as tools for faculty-student communication or for inter-student communication \cite{blessing_using_2012}. While the literature on the use of SM in the classroom is emerging, recent studies have found the platform functional for promoting concise expression of ideas, critical reading and writing skills, stronger student-teacher relationships, self-learning in an informal environment, and accountability among other benefits \cite{shiffman_twitter_2012}. The use of social media in the classroom might be viewed as a form of inquiry-based learning, an educational approach that allows the student to take ownership over the education process by self-identifying examples relevant to course curriculum, possibly outside of the classroom \cite{magnussen_impact_2000, prince_does_2004}. Benefits of similar educational techniques have been found in relation to asking students to communicate the content of a given course to a broader, general-public audience \cite{junco_effect_2011, ha_influence_2013}. However, at the same time it can be a challenge to promote active participation in this sort of activity due to students’ apprehension engaging in public discourse. Similar apprehensions at the instructor level have limited the use of Twitter as a classroom resource \cite{carpenter_how_2014}. Further, using SM in the classroom has potential disadvantages such as distracting content, overly constraining character limitations, and privacy concerns \cite{dhir_tweeters_2013}. Classes with large enrollments may deter active student engagement as has been noted in the literature \cite{ahlfeldt_measurement_2005}.  

Twitter has been employed by some in higher education as a tool for enhanced student engagement. However, it is unclear in what settings and under what circumstances Twitter can be effectively employed in the engineering classroom. In what ways can we encourage students to develop more effectively through Twitter use in the classroom?


\section*{Methodology}
Fall 2012 and Spring 2013: 290/293, 1 assignment on D2L only
Fall 2013: 290/293, treatment (n=36) and control (n=24), half D2L half Twitter
Spring 2014: 293, treatment (n=16), Twitter
Fall 2014: 292, treatment (n=46), Twitter
Spring 2015: EC only, open to all SM
Fall 2015: TBD

The tasks that students were asked to complete as part of the study involved weekly postings to Twitter relevant to the topics of discussion in the course that week. With these tasks, it was intended for the student to look beyond their standard homework and relate to the course material in a new way, independent of their textbook or course notes. Similar work by others has demonstrated success in getting students to make the connection between the classroom and the “real world” \cite{hopp_journal_2008}.

The deliverables for these tasks consisted of either a photograph or video, along with a written description, of an object or event that demonstrates the concepts relevant to the week’s course material. The students were also asked to comment on their peer's postings, thus spurring further discussion. Examples of students’ work are presented along with discussion of lessons learned through this study. Evaluation of student learning outcomes will be discussed and comparisons will be made between the small cohort and large cohort groups.

%----------------------------------------------------------------------------------------
%  REFERENCE LIST
%----------------------------------------------------------------------------------------
\vspace{4\baselineskip}\vspace{-\parskip} % Creaters proper 4 blank line spacing.
\footnotesize % Makes bibliography 10 pt font.
\bibliographystyle{unsrtnat} %Can use a different style as long as it is one which uses numbered references in the text.
\bibliography{refs}

%----------------------------------------------------------------------------------------



\end{document}
